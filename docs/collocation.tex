 \documentclass[a4paper,12pt]{article}
%\documentclass[printer]{gOMS2e}
%\usepackage{indentfirst}
%\usepackage[T1]{fontenc}
\usepackage{amsfonts}
\usepackage{amsmath,url}
\usepackage[left=2cm,right=2cm,top=2.5cm]{geometry}
\usepackage{graphicx}
\usepackage{algorithmic}
\usepackage{algorithm}

%\newcommand{\qed}{\hfill \qedbox $\quad\quad$\\[1ex]}%
%\newcommand{\beq}{\begin{equation}}
%\newcommand{\eeq}{\end{equation}}
%\newcommand{\barr}{\begin{array}}
%\newcommand{\earr}{\end{array}}
%\newcommand{\bvec}{ \left[ \!\! \barr{cccccccccccc} }
%\newcommand{\evec}{ \earr \!\! \right] }
%\newcommand{\bmat}{ \left( \!\! \barr{ccccccc} }
%\newcommand{\emat}{ \earr \!\! \right) }
%\newcommand{\e}{\mathbf{e}}
%\renewcommand{\AA}{\mathbb{A}}
%\newcommand{\BB}{\mathbb{B}}
%\newcommand{\R}{\mathbb{R}}
%\newcommand{\N}{\mathbb{N}}
%\newcommand{\EE}{{\mathbb{E}}}
%\newcommand{\LL}{{\mathcal{L}}}
%\newcommand{\FF}{{\mathbb{F}}}
%\newcommand{\XX}{{\mathbb{X}}}
%\newcommand{\Id}{\mathbb{I}}
%\newcommand{\conv}{\mathbf{conv}}
%\newcommand{\nonneg}{\mathbf{nonneg}}
%\newcommand{\quadand}{\quad\mbox{ and }\quad}

\newcommand{\dpartial}[2]{\frac{\partial {#1}}{\partial {#2}}}
\newcommand{\dtotal}[2]{\frac{d {#1}}{d {#2}}}

\newcommand{\myl}{\xi}
\newcommand{\myldot}{\myl'}
%\newcommand{\pos}{\vec{r}}
%\newcommand{\rad}{l}
\newcommand{\matr}[2]{\left[\begin{array}{#1}#2\end{array}\right]}
%\newcommand{\dpartial}[2]{\frac{\partial#1}{\partial #2}}
\newcommand{\refeq}[1]{Eq.~(\ref{#1})}
\newcommand{\refsec}[1]{Sect.~\ref{#1}}
\newcommand{\reffig}[1]{Fig.~\ref{#1}}
%\newcommand{\reftab}[1]{Table \ref{#1}}
%\usepackage{epstopdf}
%\usepackage{amsmath}
%\usepackage{amssymb}

\newcommand{\taus}[3]{\frac{\tau_{#1}-\tau_{#2}}{\tau_{#3}-\tau_{#2}}}
\newcommand{\taud}[3]{\frac{1}{\tau_{#3}-\tau_{#2}}}

\begin{document}
\sffamily
\begin{center}
\begin{LARGE}
{\bf Direct Collocation}\\
\vspace*{0.3cm}
\end{LARGE}\end{center}
\begin{center}
\begin{Large}
Greg Horn
\end{Large}

\end{center}
\vspace*{1cm}

\section{Lagrange Interpolation}
\begin{equation}
\begin{aligned}
x(\tau) &= \sum_{j=0}^D \myl_j(\tau) x_j
\end{aligned}
\label{eq:lagrange_interp_poly}
\end{equation}
where
\begin{equation}
\myl_j(\tau) = \prod_{k=0,k \neq j}^D\frac{\tau-\tau_k}{\tau_j-\tau_k}
\end{equation}
The derivative of this polynomial on an intermediate point is given by
\begin{equation}
x'(\tau) = \sum_{j=0}^D \myldot_j(\tau) x_j
\label{eq:lagrange_interp_poly_deriv}
\end{equation}

Written out for $D=3$, this looks like

\begin{align}
x(\tau) = & x_0 \taus{}{1}{0} \taus{}{2}{0} \taus{}{3}{0} + \\
          & x_1 \taus{}{0}{1} \taus{}{2}{1} \taus{}{3}{1} + \\
          & x_2 \taus{}{0}{2} \taus{}{1}{2} \taus{}{3}{2} + \\
          & x_3 \taus{}{0}{3} \taus{}{1}{3} \taus{}{2}{3}
\end{align}

By evaluating this at a certain $\tau_m$ eliminates all terms except $m=k$, giving
\begin{equation}
x(\tau_m) = x_m
\end{equation}

The derivative for $D=3$ is

\begin{align}
x'(\tau) = &x_0 (\taud{}{1}{0} \taus{}{2}{0} \taus{}{3}{0}
               + \taus{}{1}{0} \taud{}{2}{0} \taus{}{3}{0}
               + \taus{}{1}{0} \taus{}{2}{0} \taud{}{3}{0}) + \\
           &x_1 (\taud{}{0}{1} \taus{}{2}{1} \taus{}{3}{1}
               + \taus{}{0}{1} \taud{}{2}{1} \taus{}{3}{1}
               + \taus{}{0}{1} \taus{}{2}{1} \taud{}{3}{1}) + \\
           &x_2 (\taud{}{0}{2} \taus{}{1}{2} \taus{}{3}{2}
               + \taus{}{0}{2} \taud{}{1}{2} \taus{}{3}{2}
               + \taus{}{0}{2} \taus{}{1}{2} \taud{}{3}{2}) + \\
           &x_3 (\taud{}{0}{3} \taus{}{1}{3} \taus{}{2}{3}
               + \taus{}{0}{3} \taud{}{1}{3} \taus{}{2}{3}
               + \taus{}{0}{3} \taus{}{1}{3} \taud{}{2}{3})
\end{align}

evaluating this at $\tau_0$ gives:

\begin{align}
x'(\tau_0) = &x_0 (\taud{0}{1}{0} \taus{0}{2}{0} \taus{0}{3}{0}
                 + \taus{0}{1}{0} \taud{0}{2}{0} \taus{0}{3}{0}
                 + \taus{0}{1}{0} \taus{0}{2}{0} \taud{0}{3}{0}) +\\
             &x_1 (\taud{0}{0}{1} \taus{0}{2}{1} \taus{0}{3}{1}
                 + \taus{0}{0}{1} \taud{0}{2}{1} \taus{0}{3}{1}
                 + \taus{0}{0}{1} \taus{0}{2}{1} \taud{0}{3}{1}) +\\
             &x_2 (\taud{0}{0}{2} \taus{0}{1}{2} \taus{0}{3}{2}
                 + \taus{0}{0}{2} \taud{0}{1}{2} \taus{0}{3}{2}
                 + \taus{0}{0}{2} \taus{0}{1}{2} \taud{0}{3}{2}) +\\
             &x_3 (\taud{0}{0}{3} \taus{0}{1}{3} \taus{0}{2}{3}
                 + \taus{0}{0}{3} \taud{0}{1}{3} \taus{0}{2}{3}
                 + \taus{0}{0}{3} \taus{0}{1}{3} \taud{0}{2}{3})
\end{align}
which simplifies to
\begin{align}
x'(\tau_0) = &x_0 (\taud{0}{1}{0}
                 + \taud{0}{2}{0}
                 + \taud{0}{3}{0}) +\\
             &x_1 (\taud{0}{0}{1} \taus{0}{2}{1} \taus{0}{3}{1}) +\\
             &x_2 (\taud{0}{0}{2} \taus{0}{1}{2} \taus{0}{3}{2}) +\\
             &x_3 (\taud{0}{0}{3} \taus{0}{1}{3} \taus{0}{2}{3}) \\
= & x_0 C_{0,0} + x_1 C_{0,1} + x_2 C_{0,2} + x_3 C_{0,3}
\end{align}

evaluating this at $\tau_1$ gives:
\begin{align}
x'(\tau_0) = &x_0 (\taud{1}{1}{0} \taus{1}{2}{0} \taus{1}{3}{0}) +\\
             &x_1 (\taud{1}{0}{1}
                 + \taud{1}{2}{1}
                 + \taud{1}{3}{1}) +\\
             &x_2 (\taus{1}{0}{2} \taud{1}{1}{2} \taus{1}{3}{2}) +\\
             &x_3 ( \taus{1}{0}{3} \taud{1}{1}{3} \taus{1}{2}{3}) \\
= & x_0 C_{1,0} + x_1 C_{1,1} + x_2 C_{1,2} + x_3 C_{1,3}
\end{align}

evaluating this at $\tau_2$ gives:
\begin{align}
x'(\tau_2) = &x_0 (\taus{2}{1}{0} \taud{2}{2}{0} \taus{2}{3}{0}) + \\
             &x_1 (\taus{2}{0}{1} \taud{2}{2}{1} \taus{2}{3}{1}) + \\
             &x_2 (\taud{2}{0}{2}
                 + \taud{2}{1}{2}
                 + \taud{2}{3}{2}) + \\
             &x_3 (\taus{2}{0}{3} \taus{2}{1}{3} \taud{2}{2}{3}) \\
= & x_0 C_{2,0} + x_1 C_{2,1} + x_2 C_{2,2} + x_3 C_{2,3}
\end{align}

evaluating this at $\tau_3$ gives:
\begin{align}
x'(\tau_3) = &x_0 (\taus{3}{1}{0} \taus{3}{2}{0} \taud{3}{3}{0}) + \\
             &x_1 (\taus{3}{0}{1} \taus{3}{2}{1} \taud{3}{3}{1}) + \\
             &x_2 (\taus{3}{0}{2} \taus{3}{1}{2} \taud{3}{3}{2}) + \\
             &x_3 (\taud{3}{0}{3} 
                 + \taud{3}{1}{3} 
                 + \taud{3}{2}{3}) \\
= & x_0 C_{3,0} + x_1 C_{3,1} + x_2 C_{3,2} + x_3 C_{3,3}
\end{align}

The general formula for $C_{j,k}$ is
\begin{equation}
C_{j,k} =
\begin{cases}
\sum_{i=0,i\ne k}^D{\frac{1}{\tau_k-\tau_i}} & j=k \\
\frac{1}{\tau_k-\tau_j} \prod_{i=0,i\ne j,i\ne k}^D{\frac{\tau_j-\tau_i}{\tau_k-\tau_i}}  & j \ne k
\end{cases}
\end{equation}

---------------------------------------------------------------------------------------------- \\
%\begin{figure}[h]
%\includegraphics[scale=0.35]{figures/collocation_interval.eps}
%\caption{Trajectory Discretization}
%\label{fig:traj_discretization}
%\end{figure}
In direct collocation, a trajectory is broken into $N$ intervals $I_i = [t_{i,0},t_{i+1,0}], i=0,\dots N-1$ (\reffig{fig:traj_discretization}).
It is convenient to scale time on interval $I_i$ according to $t = t_{i,0} + \tau \frac{T}{N}$ with $\tau\in[0,1]$.
The differential state on interval $I_i$ is approximated as a Lagrange interpolating polynomial $\vec{x}_i^D$ of degree $D$, with $D+1$ control points $\vec{x}_{i,j}$ placed respectively at $\tau_j$:

Given an initial value $\vec{x}_{i,0}$, the model equations can be satisfied by enforcing \refeq{eq:dae} at the collocation nodes $\tau_1,\dots,\tau_D$.
This results in the collocation equations:
\begin{equation}
%0 = f\left(\sum_{k=0}^D \frac{N}{T}\left.\frac{\mathrm{d}\myl_k(\tau)}{\mathrm{d}\tau}\right|_{\tau=\tau_j} x_{i,k}, \,x_{i,j}, \,z_{i,j}, \,u_i, \,p, \,t_{i,j}\right), \; j = 1, \ldots, D.\\%, \\
0 = \vec{f}\left(\sum_{k=0}^D \frac{N}{T}\myldot_k(\tau_j) \vec{x}_{i,k}, \,\vec{x}_{i,j}, \,\vec{z}_{i,j}, \,\vec{u}_i, \,\vec{\theta}, \,t_{i,j}\right), \; j = 1, \ldots, D.%, \\
%0 &= g(x_{i,j}, \,z_{i,j}, \,u_i, \,p, \,t_{i,j}), \; j = 1, \ldots, D.
\label{eq:coll_eqns}
\end{equation}
When \refeq{eq:coll_eqns} is satisfied, the final value $\vec{x}_{i+1,0}$ can be recovered by evaluating \refeq{eq:lagrange_interp_poly} at $\tau=1$:
\begin{equation}
\vec{x}_{i+1,0} = \vec{x}_i^D(t_{i+1,0}) = \sum_{j=0}^D \myl_j(1) \vec{x}_{i,j}.
\label{eq:continuity_eqs}
\end{equation}

The collocation points $\tau_j$ must be chosen as the roots of shifted Gauss-Jacobi polynomials so that \refeq{eq:continuity_eqs} is an accurate Gauss quadrature integration \cite{Biegler2010}.%and an accurate implicit Runge-Kutta method \cite{Biegler2010}.
The special Gauss-Jacobi polynomials Gauss-Legendre or Gauss-Radau are often used for their A-stability and for their high-order accuracy.
Numerical values for these roots can be found in \cite{Biegler2010}, though it is convenient to use the SciPy function scipy.special.js\_roots \cite{Scipy2001}.

We summarize \refeq{eq:coll_eqns} and \refeq{eq:continuity_eqs} with $X_i = (\vec{x}_{i,1},\dots,\vec{x}_{i,D})$, $Z_i = (\vec{z}_{i,1},\dots,\vec{z}_{i,D})$, $i=0,\dots,N-1$, $\vec{x}_i=\vec{x}_{i,0}$, $i=0,\dots,N$, as
\begin{equation}
\begin{aligned}
%x_i &\equiv x_{i,0} \\
%X_i &= (x_{i,1}^T, \dots, x_{i,D}^T)^T \\
%Z_i &= (z_{i,1}^T, \dots, z_{i,D}^T)^T \\
0 &= \vec{G}(X_i,Z_i,\vec{u}_i,\vec{\theta},T) \\
\vec{x}_{i+1} &= \phi(\vec{x}_i,X_i).
\label{eq:coll_eqns_summary}
\end{aligned}
\end{equation}

\subsection{Quadrature States}
\label{sec:quadrature_states}
In \refsec{sec:ocp_statement} it was stated that some integral terms such as \refeq{eq:mean_power_objective_function} can be evaluated by adding a differential state to the problem and evaluating it at $T$.
If this integral term is used only in the cost function, it can be beneficial to evaluate it without adding an additional state to the system.

Consider the problem where some derivative $\dot{q}$ is known at the collocation nodes, and $q(\tau=1)$ should be computed by assuming that $q(\tau=0)=0$ and integrating over one collocation interval.
Writing out \refeq{eq:lagrange_interp_poly_deriv} at the collocation nodes:
\begin{equation}
\matr{c}{
\dot{q}(\tau_1) \\
\vdots   \\
\dot{q}(\tau_D)}
=
\frac{N}{T}
\matr{ccc}{
\myldot_1(\tau_1) & \cdots & \myldot_D(\tau_1) \\
\vdots &  \ddots & \vdots \\
\myldot_1(\tau_D) & \cdots & \myldot_D(\tau_D)}
%
\matr{c}{
q(\tau_1) \\
\vdots   \\
q(\tau_D)},
\end{equation}
and solving for $q(\tau_j)$ yields:
\begin{equation}
\matr{c}{
q(\tau_1) \\
\vdots   \\
q(\tau_D)}
%
=
%
\frac{T}{N}
\matr{ccc}{
\myldot_1(\tau_1) & \cdots & \myldot_D(\tau_1) \\
\vdots           & \ddots & \vdots \\
\myldot_1(\tau_D) & \cdots & \myldot_D(\tau_D)}^{-1}
%
\matr{c}{
\dot{q}(\tau_1) \\
\vdots   \\
\dot{q}(\tau_D)}.
\end{equation}
Combining this with \refeq{eq:continuity_eqs} yields:
\begin{equation}
\begin{aligned}
%q(\tau=1) &= \matr{ccc}{\myl_1(1) & \cdots & \myl_D(1)}\matr{c}{q(\tau_1) \\ \vdots \\ q(\tau_D)} \\
q(\tau=1) &=
\matr{ccc}{\myl_1(1) & \cdots & \myl_D(1)}
\frac{T}{N}
\matr{ccc}{
\myldot_1(\tau_1) & \cdots & \myldot_D(\tau_1) \\
\vdots           & \ddots & \vdots \\
\myldot_1(\tau_D) & \cdots & \myldot_D(\tau_D)}^{-1}
%
\matr{c}{
\dot{q}(\tau_1) \\
\vdots   \\
\dot{q}(\tau_D)} \\
&=
\frac{T}{N}\Lambda^T
\matr{c}{
\dot{q}(\tau_1) \\
\vdots   \\
\dot{q}(\tau_D)},
\end{aligned}
\end{equation}
%
where $\Lambda$ is a constant vector since both $\myl_j(\tau_k)$ and $\myldot_j(\tau_k)$ are constant.
Integrating over all collocation intervals yields the value at $T$:
\begin{equation}
q(T) = \frac{T}{N}\Lambda^T
\sum_{i=0}^{N-1}
\matr{c}{
\dot{q}(t_{i,1}) \\
\vdots   \\
\dot{q}(t_{i,D})}
\end{equation}
%
so the integral term of the cost function from \refeq{eq:general_continuous_ocp} can be computed as:
\begin{equation}
\int_0^T \! J_\mathrm{L}(\vec{x}(t),\vec{z}(t),\vec{u}(t),\vec{\theta},t,T) \, \mathrm{d}t =
\frac{T}{N}\Lambda^T
\sum_{i=0}^{N-1}
\matr{c}{
J_\mathrm{L}(\vec{x}_{i,1},\vec{z}_{i,1},\vec{u}_i,\vec{\theta},t_{i,1},T) \\
\vdots   \\
J_\mathrm{L}(\vec{x}_{i,D},\vec{z}_{i,D},\vec{u}_i,\vec{\theta},t_{i,D},T)}.
\end{equation}

\end{document}
